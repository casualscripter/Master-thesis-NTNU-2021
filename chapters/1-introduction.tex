\chapter{Introduction}

The experience of the last years has shown that the possibility to see a system under investigation through the eyes of the suspect can have many advantages in terms of investigation.

The author has developed a system based on open source that should enable all parties involved in the investigation process to safely virtualize a seized computer.

\section{Target groups}

The target groups for the system are cybercrime investigators, digital forensics experts, and IT specialists in the Lower Saxony Police.

The cybercrime investigator is the first who knows about the case. Because he follows the case from start to finish, he knows it best. He plans all the individual steps. He is the contact person for the public prosecutor's office. He plans the searches and accompanies them as part of the identification phase of the digital forensics process. \cite{Aarnes2017:16} % page 16
He is doing the police questioning of attestors and suspects. He obtains additional information from external parties. He makes forensic investigation requests to the digital forensic experts. He lets explain complicated things to him by IT specialists. He writes the final report, which is understandable for laymen.

The digital forensics expert often only gets to see the forensic investigation request and the evidence itself. At the crime scene, he is responsible for doing live forensics if needed and seizing the storage devices as part of the collection phase of the digital forensics process. He is collection digital raw data by copying the source secured by write blocking technology in a forensically sound manner. \cite{Aarnes2017:16} % page 16 (and 23)
He preserves the chain of custody and evidence integrity among other things, through the documentation incl. serial numbers of storage devices, photos, geolocation, hash values, timestamp service and digital signatures from the very start. \cite{Aarnes2017:23} % page (16 and)  23
At Lower Sacony the most common forensic image format is EWF\_E01.
He writes a report that can be used in court about the presence or absence of relevant digital traces on the storage devices.

The IT specialist explains complicated technical issues to the cybercrime investigator. If necessary, he explains to the cybercrime investigator the connections between the results of the forensic investigation and the case. He analyzes data, such as network traffic, mass data, software. He develops individual software solutions.

\section{Problems}

A forensic image format is supposed to protect against changes. However, an installed operating system still wants to be able to write to the disk. The differences between real and virtualized hardware require different drivers. Some systems even require special hardware to boot. Some systems cannot be virtualized, but have to be emulated. Authentication has to be overcome. Some operating systems require an individual recipe.

Not all target groups know about the problems. Not all target groups can identify the problems. Not all target groups know how to solve the problems. Not all target groups have the technical requirements. Not all target groups can solve the problems technically.

\section{Goal}

% warum?

All target groups, regardless of their technical equipment and previous training, should be able to virtualize forensic images until they have successfully logged on to the system.
A graphical user interface should support cybercrime investigators in particular.
The cybercrime investigator is enabled to assist in the Examination phase. \cite{Aarnes2017:34} % page 34
He could give the digital forensic expert a hint where to search deeper.
The cybercrime investigator has to be able to verify hash values, digital signatures, and timestamps of a forensic image prior to examination or analysis phase of the digital forensics process.
Virtualisation could also be usefull while in the presentation phase of the digital forensic process at the court as some kind of visualisation. \cite{Aarnes2017:46} % page 46
In contrast to propretary solutions, the focus is not only on the mainstream (Microsoft Windows 10). Older Windows versions (XP, Vista, 7 and 8) are also to be taken into account. The virtualization of macOS guest systems under non-macOS host systems should also be supported. GNU/Linux, BSD and Unix should also be supported. Even embedded devices (e.g. QNX) and IoT (e.g. Contiki OS) should be supported.
If a system or platform is not yet supported, it should be possible to extend the system easily.

\section{Outcome}

% wichtig?

The system enables cybercrime investigators to prioritize the storage devices to be examined for forensic evaluation as part of preview and triage.
During the police questioning of a crime suspect, he can even be confronted with his own system without danger of data loss.
It helps the digital forensics expert to better understand the case. The following better structured approach will save time.
The IT specialist can gain additional data (eg network traffic, memory, ...) for the analysis through a running system.
Since the system can be used by everyone independently and there are also no more unnecessary large amounts of data due to the export of virtual machines, everyone involved can work on a case in parallel.

\section{Research question(s)}

Main Research Question 1:\newline
\noindent Is it possible to develop a system with existing open source software projects with which even a technically less experienced investigator can virtualise and enter a forensic image in a forensically sound manner?\newline

\noindent Research Question 2:\newline
\noindent What are the disadvantages of such a product compared to proprietary solutions like Sumuri Carbon VFS, GetData Forensic Explorer or Arsenal Image Mounter?\newline

\noindent Research Question 3:\newline
\noindent What are the advantages of such a product compared to proprietary solutions like Sumuri Carbon VFS, GetData Forensic Explorer or Arsenal Image Mounter?

\section{Construction of the master thesis}

TODO

\subsection{Introduction}

[...]

\subsection{Requirements}

[...]

\subsection{Technical design}

[...]

\subsection{Development Process}

[...]

\subsection{Implementation}

[...]

\subsection{Deployment}

[...]

\subsection{Testing and user feedback}

[...]

\subsection{Discussion}

[...]

\subsection{Conclusion}

[...]

\subsection{Future work}

[...]
