\chapter*{Zusammenfassung}

Für einen teschnisch weniger versierten Ermittler ist es von Vorteil, wenn er das System des Beschuldigten virtualisieren kann. Es mit den Augen des Beschuldigten sehen zu können ist einfacher, als komplizierte, integrierte forensische Werkzeuge zu nutzen.

Basiert die Lösung auf Open Source, kann sogar Geld gespart werden.

Im theoretischen Teil dieser Masterarbeit wurden lediglich die notwendigen Anforderungen zusammengetragen. Auch die Einführung konnte nur theoretisch betrachtet werden. Ergänzende Anforderungen wurden nicht verworfen, sondern im Ausblick (Future work) gesammelt.

Im praktischen Teil wurde eine Lösung ohne individuelle grafische Oberfläche entwickelt. Die Bedienung basiert auf dem Kontextmenü des Dateimanagers. Da es sich um ein komplettes bootbares GNU/Linux System handelt, entfällt für den Anwender die fehleranfällige Installation aller einzelnen Bestandteile.

Der Vergleich mit proprietären Lösungen hat ergeben, dass eine individuelle grafische Oberfläche fehlt. Sie würde die Bedienung vereinfachen und den Anwender besser durch den Workflow führen.

Die immer wieder durchgeführten Tests haben jedoch auch aufgezeigt, dass das Endprodukt sehr flexibel ist und Vorteile im Bezug auf Betriebssysteme hat, die nicht so weit verbreitet sind.