\chapter{Development process}
\label{chap:devproc}

\section{Theory}

Prior to his employment, the author worked for several years in software development at the Central Police Department of the Lower Saxony Police, including as an administrator, developer, requirements analysis officer and project manager in projects with five to over 80 project members.

As a rule, the waterfall model with several iterations was used.

In the case of particularly important completed projects, further development was continued in follow-up projects.

Since different projects had to be worked on at the same time, individual employees or entire teams were regularly involved in several projects.

The office was structured in a matrix organisation, in which there were separate teams for requirements management, requirements analysis, server-side development, client-side development, final development of graphical interfaces, database modelling, test laboratory, implementation management including the creation of manuals, public relations, etc.

Communication and discipline were another important cornerstone for successful projects.

\section{Practice}

For the preparation of a practical master thesis as a single person in their spare time, access to a comparable structure was not possible.

One person has to carry out all tasks alone. Communication with other project members is not possible or does not exist. It is also not possible to work uninterruptedly on the project for weeks at a time, as work and family have to be attended to in parallel.

It has proven practical to break down the big picture into smaller parts, as is done in Extreme Programming, in order to be able to use the free time as effectively as possible.

\section{Versioning}

A Gitea repository is used for the development: \url{https://git.neumannsland.de/casualscripter/Masterthesis}